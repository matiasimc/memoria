\chapter{Propuesta}

En este trabajo se propone realizar la implementación en Dart de un sistema de inferencia de facetas de declasificación, que incluya el análisis de \textit{Type-based Relaxed Noninterference}, mediante la realización de un plugin para Dart Analyzer. En este capítulo se detallan los problemas de inferencia a resolver, las estrategias utilizadas para resolverlos, los cambios al trabajo original y el subconjunto del lenguaje soportado.

\section{Problema de inferencia}
Se discute el problema de inferencia a resolver y ejemplos de lo que se busca del sistema a implementar.

\section{Consideraciones de diseño}
Se discuten las alternativas disponibles respecto a la decisión sobre las facetas de métodos que pertenecen al core del lenguaje, y por qué Bot -> Bot es la escogida. Explicar que de todas formas sería un parámetro configurable de la herramienta.

\section{Generación de constraints de subtyping}
Se muestra en palabras la generación de constraints para las distintas expresiones.

\section{Resolución de constraints}
Se explica el uso de las operaciones \texttt{join} y \texttt{meet} sobre la lattice, simplificación de constraints y otros.

\section{Extensión de propuesta teórica}
Mostrar los cambios y extensiones necesarias para ajustarse a un lenguaje como Dart y el subconjunto soportado.

\section{Interacción con el usuario}
Cuál es la interacción con el usuario deseada.
