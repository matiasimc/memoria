\chapter{Propuesta}

En este trabajo se propone realizar la implementación del análisis de \textit{Type-based Relaxed Noninterference} en Dart, y una herramienta de inferencia de facetas de declasificación, mediante la realización de un plugin para Dart Analyzer. En este capítulo se detallan los problemas de inferencia a resolver, las estrategias utilizadas para resolverlos y el subconjunto del lenguaje soportado.

\section{DRAFT de ejemplos}

Inferencia completa basado en uso de distintas variables.
 \begin{lstlisting}
int foo(String a, String b) {
  return a.length + b.indexOf("a");
}
->
@IntPlus int foo(@StringLength String a, @StringIndexOf String b) {
  return a.length + b.indexOf("a");
}
 \end{lstlisting}

¿Qué pasa si se llama a un metodo que retorna High?

 \begin{lstlisting}
class int {
  @High int operator +(int b);
  ...
}

int foo(String a, String b) {
  return a.length + b.indexOf("a");
}
->
@High int foo(@StringLength String a, @StringIndexOf String b) {
  return a.length + b.indexOf("a");
}
 \end{lstlisting}

 Inferencia en condicionales.

 \begin{lstlisting}
int foo(bool a) {
 bool b = !a;
 int ret = 0;
 if (a) {
   ret = 4;
 }
 else {
   ret = 5;
 }
 return ret;
}

@High int foo(@BoolNot bool a) {
 @High bool b = !a;
 @High int ret = 0;
 if (a) {
   ret = 4;
 }
 else {
   ret = 5;
 }
 return ret;
}
 \end{lstlisting}

  Si la variable retornada no tiene ninguna llamada a método, retornará High. En este caso las asignaciones son válidas, pues se puede asignar cualquier cosa a High (si los tipos de implementación son válidos).

  \begin{lstlisting}
 @High int foo(@Bool bool a) {
  @Int int ret = 0;
  if (a) {
    ret = 4;
  }
  else {
    ret = 5;
  }
  return ret;
 }
  \end{lstlisting}

  Si forzamos la declaración de \texttt{ret} como pública, la asignación será inválida, puesto que el contexto de seguridad del \texttt{if} es \texttt{BoolNot}, lo cual no tiene una relación de subtyping válida con \texttt{int} para realizar la asignación.
