\chapter{Propuesta}

En este trabajo se propone realizar la implementación en Dart de un sistema de inferencia de facetas de declasificación, que incluya el análisis de \textit{Type-based Relaxed Noninterference}, mediante la realización de un plugin para entornos de desarrollo integrado (IDE). En este capítulo se detallan los problemas de inferencia a resolver, las estrategias utilizadas para resolverlos, los cambios al trabajo original y el subconjunto del lenguaje soportado.

\section{Problema de inferencia}
Se discute el problema de inferencia a resolver y ejemplos de lo que se busca del sistema a implementar.

\section{Consideraciones de diseño}
Se discuten las alternativas disponibles respecto a la decisión sobre las facetas de métodos que pertenecen al core del lenguaje, y por qué Bot -> Bot es la escogida. Explicar que de todas formas sería un parámetro configurable de la herramienta.

\section{Generación de constraints de subtyping}
Se muestra en palabras la generación de constraints para las distintas expresiones.
\subsection{Declaración de métodos}
\subsection{Llamadas a métodos}
\subsubsection{Retorno}
\subsubsection{Argumentos}
\subsubsection{Encadenamiento de llamados}
\subsection{Expresión de retorno}
\subsection{Declaración y asignación a variables}
\subsection{Expresiones condicionales}



\section{Resolución de constraints}

\subsection{Simplificación y eliminación de constraints}
\subsection{Agrupación de constraints}
\subsubsection{Join}
\subsubsection{Meet}
\subsection{Unificación y substitución}


\section{Extensión de propuesta teórica}
Mostrar los cambios y extensiones necesarias al trabajo de type-based declassification para ajustarse a un lenguaje como Dart, y el subconjunto de Dart soportado.

\section{Interacción con el usuario}
Cuál es la interacción con el usuario deseada.
