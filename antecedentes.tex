\chapter{Antecedentes}

En este capítulo se discuten los antecedentes y el marco teórico necesario para poder entender este trabajo. Además, se muestran las definiciones formales utilizadas en el resto del documento.

\section{Protección de confidencialidad en computación}
Se nombran y caracterizan brevemente distintas técnicas: criptografía, control de acceso, language-based security (LBS). Por qué LBS es más expresivo y/o efectivo.

\subsection{Tipado de seguridad y control de flujo}
Se explica en que consiste, se muestra la lattice de dos niveles de seguridad y se introduce el concepto de information flow control.
\subsubsection{Flujo explicito}
Se muestra un ejemplo de flujo explicito.
\subsubsection{Flujo implicito}
Se muestra un ejemplo de flujo implicito.
\subsubsection{Trabajos relacionados}
Se muestran distintas herramientas de tipado de seguridad, como Jif.
\subsection{No-interferencia}
Se formaliza la propiedad de confidencialidad con noninterference. Se muestra un ejemplo, o se hace referencia a ejemplos anteriores.

\subsection{Declasificación}
Se explica por qué es necesario tener un mecanismo de declasificación, y cómo se puede controlar. Aquí se da el pase-gol a TYPE-BASED DECLASSIFICATION.
\section{Type-based declassification}
Breve introducción.
\subsection{Tipos de dos facetas}
\subsection{Sistema de tipos}
\subsection{Propiedades}
Safe, relaxed noninterference

\section{Inferencia de tipos}
\subsection{Objetivo y usos}
\subsection{Constraints}
\subsection{Unificación}
\subsection{Decidibilidad}

\subsection{Inferencia de tipos de seguridad}
Mencionar por qué es posible y trabajos relacionados al respecto. Explicar y ejemplificar el uso de PC.

\section{Lenguaje Dart}
Qué es Dart, para qué se usa, cuáles son sus características principales y por qué fue escogido.
\subsection{Dart Analyzer}
\subsection{Analyzer Plugin}
\subsection{Otras herramientas}
Se menciona librería de unit test, manejo de dependencias y versión de Dart.
