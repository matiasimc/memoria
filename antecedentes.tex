\chapter{Antecedentes}

En este capítulo se discuten los antecedentes y el marco teórico necesario para poder entender este trabajo. Además, se muestran las definiciones formales utilizadas en el resto del documento.

\section{Protección de confidencialidad en computación}

\subsection{No-interferencia}
Los sistemas seguros utilizan el concepto de no-interferencia (\textit{noninterference})~\cite{noninterference} para referirse al cumplimiento de políticas de seguridad.

Si consideramos que un sistema maneja dos niveles de seguridad, \textit{low} o de baja confidencialidad, y \textit{high} o de alta confidencialidad, tenemos la siguiente definición de noninterference:

\begin{defn}[Noninterference]
  Un sistema cumple con la propiedad de noninterference si y solo las acciones realizadas por entes \textit{high} no tienen efecto en lo que entes \textit{low} pueden ver.
\end{defn}

\subsection{Language-based security}
Los sistemas computacionales abordan la protección de confidencialidad con diversas técnicas, tales como criptografía y control de acceso. Ninguna de estas técnicas garantiza efectivamente la protección de confidencialidad. En efecto, los esquemas de encriptación tienen que ser actualizados constantemente debido al descubrimiento de vulnerabilidades, y control de acceso no restringe la propagación de información~\cite{myers-phd}.

Existe un conjunto de técnicas más efectivas para fortalecer la seguridad de un sistema, utilizando propiedades de los lenguajes de programación. A este conjunto se le denomina \textit{Language-based security} (LBS).

\subsection{Tipado de seguridad y control de flujo}
Una de las técnicas más efectivas de LBS con análisis estático es \textit{tipado de seguridad} en un \textit{lenguaje de seguridad}. En un lenguaje de seguridad, los valores y los tipos son anotados con niveles de seguridad para clasificar la información que el programa manipula. Dichos niveles de seguridad forman una \textit{lattice}\footnote{Un orden parcial, donde todo par de elementos tiene un único supremo e ínfimo}.

	\begin{figure}[ht]
		\centering
		\begin{tikzpicture}
			\node(H) 												{$H$};
			\node(L)      [below of=H]       {$L$};
			\draw(H)      -- (L);
		\end{tikzpicture}
		\caption{\textit{lattice} de dos niveles de seguridad}
	\end{figure}


  	Por ejemplo con la \textit{lattice} de dos niveles de seguridad $L \sqsubseteq H$, se puede distinguir entre valores enteros públicos o de baja confidencialidad ($Int_L$) y valores enteros privados o de alta confidencialidad ($Int_H$). El sistema de tipos usa estos niveles de seguridad para prevenir que la información confidencial no fluya directa o indirectamente hacia canales públicos~\cite{volpanoAl:S96}, técnica que en general se denomina \textit{information flow control}.

  	\begin{lstlisting}
  String @H login(String @L guess, String @H password) {
  	if (password == login) return "Login successful";
  	else return "Login failed";
  }
  	\end{lstlisting}

    En el código anterior, los tipos de los parámetros y el tipo de retorno de la función están etiquetados con niveles de seguridad.

\subsubsection{Flujo explicito}
Se denomina flujo explícito a las instrucciones del programa que directamente asignan un valor a una variable con distintos niveles de seguridad. El siguiente programa ilustra el flujo explícito.

\begin{lstlisting}
void @L foo(int @H highVar, int @L lowVar) {
  int @H v1 = lowVar;
  int @L v2 = highVar;
}
\end{lstlisting}

En el ejemplo, la primera asignación no representa un riesgo de seguridad, puesto que se asigna un valor público a una variable confidencial. Por otra parte, la segunda asignación es insegura, debido a que se asigna un valor confidencial a una variable pública, que luego puede ser utilizada en contextos no deseados.

\subsubsection{Flujo implicito}
Se denomina flujo implícito a las instrucciones del programa que indirectamente dan conocimiento de algún aspecto de una variable, usualmente mediante instrucciones condicionales. Podemos adaptar el mismo programa de \texttt{login} visto anteriormente para ilustrar el flujo implícito.

\begin{lstlisting}
  String @L login(String @L guess, String @H password) {
    String @L ret;
    if (password == login) ret = "Login successful";
    else ret = "Login failed";
    return ret;
  }
\end{lstlisting}

En este ejemplo, las instrucciones de asignación a la variable \texttt{ret} son seguras por si solas, pero no lo son considerando que su ejecución depende del valor de una variable confidencial, en este caso \texttt{password}.

\subsection{Declasificación}
Noninterference se considera una propiedad muy estricta, debido a que aplicaciones reales y prácticas la vulneran fácilmente. En efecto, el programa de \texttt{login} es un buen ejemplo.

\begin{lstlisting}
  String @H login(String @L guess, String @H password) {
    if (password == login) return "Login successful";
    else return "Login failed";
  }
\end{lstlisting}

Este programa no cumple con noninterference, debido a que el adversario puede aprender sobre la variable confidencial \texttt{password} observando el valor de retorno del método para distintas ejecuciones.

Sin embargo, deseamos que el programa anterior sea aceptado a pesar de violar noninterference, pues de otra forma no tendríamos cómo realizar la autenticación. Para solucionar este problema, los lenguajes de seguridad adicionan mecanismos para \textit{declasificar} la información confidencial, implementados de diferentes formas~\cite{sabelfeldSands:JCS09}. Una de ellas, por ejemplo en Jif (un lenguaje de seguridad)~\cite{jif} es usar un operador \texttt{declassify}, como se indica en el siguiente ejemplo, declasificando la comparación de igualdad del parámetro confidencial \texttt{password} con el parámetro público \texttt{guess}

\begin{lstlisting}
  String login(String password, String guess) {
    if (declassify(password == guess)) return "Login Successful";
    else return "Login failed";
  }
\end{lstlisting}

Esto no corresponde a una amenaza de seguridad, debido a que el resultado de la operación de comparación es negligible con respecto al parámetro privado \texttt{password}. Sin embargo, usos arbitrarios del operador \texttt{declassify} pueden resultar en serias fugas de información, como por ejemplo \texttt{declassify(password)}.
\section{Type-based declassification}
Varios mecanismos se han explorado para controlar el uso de declasificación, y poder asegurar además una propiedad de seguridad para el programa~\cite{sabelfeldSands:JCS09}. En esta dirección, Cruz et al.~\cite{cruzAl:ecoop2017} recientemente propusieron \textit{type-based declassification} como un mecanismo de declasificación que conecta la abstracción de tipos con una forma controlada de declasificación, en una manera intuitiva y expresiva, proveyendo garantías formales sobre la seguridad del programa.

En \textit{type-based declassification} los tipos tienen dos facetas; la faceta privada, que refleja el tipo de implementación, y la faceta pública, que refleja las operaciones de declasificación sobre los valores de dicho tipo. Por ejemplo, el tipo $\text{StringEq} \triangleq [\text{eq} : \text{String} \rightarrow \text{Bool}]$ autoriza la operación \texttt{eq} sobre un String. Entonces se puede usar el tipo de dos facetas $\text{String} < \text{StringEq}$, en donde String es un subtipo de StringEq, para controlar la operación de declasificación de la igualdad sobre \texttt{password}.

\begin{lstlisting}
  String<String login(String<StringEq password, String<String guess) {
  	if (password.eq(guess)) return "Login successful";
  	else return "Login failed";
  }
\end{lstlisting}

Al igual que en tipado de seguridad con etiquetas (@L y @H), la faceta de declasificación es parte de la jerarquía de tipos, la que forma una \textit{lattice} como se muestra en la figura \ref{latticeTBD}. Si la faceta pública coincide con la faceta privada, entonces toda operación sobre el valor estará autorizada. Cuando esto sucede, se refiere usualmente a la faceta pública con \texttt{Bot}, por encontrarse siempre en la parte inferior de la \textit{lattice}.

	\begin{figure}[ht]
		\centering
		\begin{tikzpicture}[node distance=2cm]
			\node(Top) 												{$Top$};
			\node(StringEq)		[below right of=Top]			{$StringEq$};
			\node(StringHashEq)      [below of=StringEq]       {$StringHashEq$};
			\node(String)				[below of=StringHashEq]       {$String$};
			\node(int)					[below left of=Top] 			{$int$};
			\draw(Top)      -- (StringEq);
			\draw(Top)      -- (int);
			\draw(StringEq)      -- (StringHashEq);
			\draw(StringHashEq)      -- (String);
		\end{tikzpicture}
    \label{latticeTBD}
		\caption{\textit{lattice} de tipos de dos facetas}
	\end{figure}

Cuando se quiere referir a una faceta pública vacía o que no autoriza ninguna operación, se usa \texttt{Top}, por encontrarse en la parte superior de la \textit{lattice}.

En estricto rigor, los métodos declarados en la faceta pública también poseen tipos de dos facetas en sus firmas. Así, el tipo StringEq visto anteriormente se define como  $\text{StringEq} \triangleq [\text{eq} : \text{String<String} \rightarrow \text{Bool<Bool}]$.

Existen dos reglas principales para comprobar que un programa con facetas de declasificación se encuentra bien tipado. En primer lugar, la llamada a un método sobre un valor cuya faceta pública autoriza la operación, retorna la faceta pública que haya sido declarada como retorno para aquella operación. Por ejemplo, si tenemos el tipo $\text{StringHashEq} \triangleq [\text{hash} : \text{String<String} \rightarrow \text{String<StringEq}]$ y llamamos al método $\text{hash}$ sobre un valor que declara la faceta pública $\text{StringHashEq}$, la faceta de retorno de esa llamada será $\text{String<StringEq}$.

La segunda regla expresa que la llamada a un método sobre un valor cuya faceta pública no autoriza la operación, retorna \texttt{Top}. Esto ocurre, por ejemplo, si llamamos al método $\text{hash}$ sobre un valor que declara la faceta pública $\text{StringEq}$.

La propiedad de seguridad que se demuestra para el sistema de tipos de \textit{type-based declassification} es una forma de noninterference con políticas de declasificación, denominada \textit{Relaxed noninterference}. Un lenguaje de seguridad que cumple esta propiedad, garantiza que la información confidencial sólo puede fluir hacia canales públicos de forma controlada, por medio de las políticas de declasificación.

\section{Inferencia de tipos}
\subsection{Objetivo y usos}
\subsection{Constraints}
\subsection{Unificación}
\subsection{Decidibilidad}

\subsection{Inferencia de tipos de seguridad}
Mencionar por qué es posible y trabajos relacionados al respecto. Explicar y ejemplificar el uso de PC.
