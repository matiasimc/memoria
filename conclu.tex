\begin{conclusion}

	La desclasificación basada en tipos muestra una conexión entre las relaciones de subtipos de un lenguaje orientado a objetos, y las relaciones de orden que conforman los tipos de seguridad, para proponer un sistema de tipos que cumple una versión relajada de no-interferencia. Con esta propuesta, Cruz \textit{et al.} abordan parcialmente el desafió de integrar los modelos de control de flujo de información con infraestructuras existentes. Este trabajo materializa aquella propuesta, con una implementación para un subconjunto del lenguaje Dart, en conjunto con un sistema de inferencia y una extensión para ambientes de desarrollo.

	A pesar del foco de seguridad que tiene un trabajo de estas características, la formulación del problema de inferencia y el uso de la extensión para integrar los resultados fueron concebidos teniendo al programador en mente, para facilitarle el trabajo y mejorar su experiencia programando, entre otros beneficios. Esta experiencia puede mejorar aún más, agregando nuevas características a la extensión.

	\section*{Trabajo futuro}

	\paragraph{Formalización de inferencia.}En este trabajo se implementó un sistema de inferencia sin demostrar formalmente las propiedades que cumple. Por ejemplo, es deseable demostrar que el sistema propuesto siempre infiere las facetas públicas más ajustadas al uso de las expresiones, o que el sistema de inferencia preserva la propiedad de no-interferencia relajada. En este sentido, es deseable la formalización de la inferencia de tipos de dos facetas, antes de realizar cualquier extensión a este trabajo, cuyo objetivo fue demostrar el uso práctico del enfoque propuesto por Cruz \textit{et al.} Además, es deseable realizar un análisis de la complejidad de los algoritmos propuestos.

	\paragraph{Extensión al subconjunto soportado.}Este trabajo soporta un subconjunto pequeño del lenguaje Dart, lo que no permite probarlo en aplicaciones reales de mayor envergadura. Por ejemplo, la implementación actual no soporta funciones y variables definidas de forma global, excepciones, funciones anónimas, entre otros. Soportar características avanzadas del lenguaje, e implementar la herramienta en otros lenguajes de programación, permitiría posicionar a la herramienta como una alternativa competente de análisis de control de flujo para aplicaciones en producción.

	\paragraph{Características de la extensión para ambientes de desarrollo.}La extensión para ambientes de desarrollo implementada en este trabajo solo muestra los resultados de la inferencia, pero no permite al programador tomar acciones automáticas al respecto. Por ejemplo, sería posible asistir al usuario en la definición de una faceta pública que ha sido declarada, navegar al lugar donde se define una faceta pública al ubicarse en la faceta declarada, definir y declarar una faceta pública basándose en el resultado de la inferencia, entre otros.

	\paragraph{Extensión a polimorfismo.}Una característica interesante a considerar en una posible extensión a este trabajo es el polimorfismo paramétrico. Con esto, es posible la definición de estructuras de datos paramétricas en tipos de dos facetas, lo que implica tener polimorfismo en facetas públicas. Además, se pueden definir facetas públicas polimórficas. Por ejemplo, la faceta pública $\mathtt{Eq[X]}\triangleq [\mathtt{eq} : \mathtt{X} \rightarrow \mathtt{Bool}]$ está parametrizada por un tipo \texttt{X}.


\end{conclusion}
