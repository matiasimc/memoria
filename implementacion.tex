\chapter{Implementación}
En esta sección se detalla la implementación de este trabajo, que se dividió en dos componentes principales. Primero, se implementó un sistema de inferencia para type-based declassification. Segundo, se elaboró un plugin para los editores de texto más populares que integra el resultado de la inferencia.

\section{Implementación de sistema de inferencia}

\subsection{Diagrama de componentes y descripción general}
Se explica el funcionamiento general de la inferencia.

\subsection{Dart Analyzer}
Explicar funcionalidades utilizadas de la librería Dart Analyzer. Información contenida en AST, en especial inferencia de tipos. Limitaciones (si es que tiene alguna relevante).

\subsection{Representación de tipos}
Cómo se representaron los distintos tipos (type variables, arrow types, object types, Top, Bot, OrType).

\subsection{Representación de constraints}
En qué consisten las constraint de subtyping implementadas.

\subsection{Representación de facetas de declasificación}
Uso de anotaciones para indicar las facetas. Abstract classes de Dart en archivo sec.dart para declararlas. Mencionar el parsing de facetas declaradas a object types.

\subsection{Almacenamiento de información relevante}
Uso de diccionarios para llevar registro del tipo de ciertos elementos o expresiones.

\subsection{Fase de generación de constraints}
Uso del patrón visitor para recorrer el AST.

\subsubsection{Recolección de errores}

\subsection{Fase de resolución de constraints}

\subsubsection{Recolección de errores}

\subsection{Testing}
Cómo se testea la inferencia.

\section{Implementación de plugin}

\subsection{Diagrama de componentes y descripción general}
Explicar funcionamiento general e integración con sistema de inferencia.

\subsection{Configuración inicial}
Uso de la herramienta y creación del archivo sec.dart en primera ejecución del análisis.

\subsection{Tipos de errores e información}
