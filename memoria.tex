\documentclass[upright, contnum]{umemoria}

%fix for the oneside argument
\makeatletter
\g@addto@macro\titlepage{\pagenumbering{Alph}}
\g@addto@macro\endtitlepage{\pagenumbering{roman}}
\makeatother

\depto{Departamento de Ciencias de la Computación}
\author{Matías Ignacio Meneses Cortés}
\title{Type-Based Declassification en Dart: Implementación y Elaboración de Herramientas de Inferencia}
\auspicio{}
\date{Abril 2018}
\guia{Éric Tanter}
\carrera{Ingeniero Civil en Computación}
\memoria{Memoria para optar al Título de \break  Ingeniero Civil en Computación}
\comision{}

\usepackage{lipsum}

\usepackage[utf8]{inputenc}
\usepackage[T1]{fontenc}
\usepackage{color}

\definecolor{dkgreen}{rgb}{0,0.6,0}
\definecolor{gray}{rgb}{0.5,0.5,0.5}
\definecolor{mauve}{rgb}{0.58,0,0.82}

\usepackage{listings}
\lstset{%
	basicstyle=\large,
	numberstyle=\tiny,
	numbersep=15pt,tabsize=4,
	flexiblecolumns=true,
	keywordstyle=\color{blue},
	commentstyle=\color{dkgreen},
	stringstyle=\color{mauve},
	numberstyle=\tiny\color{gray},
	language=Java,
	breaklines=true,
	breakatwhitespace=true,
  showstringspaces=false,
  aboveskip=1.2em,
  belowskip=1.2em,
	morekeywords={*,num,String,var,library,get,set} ,
}

\begin{document}
\frontmatter
\maketitle

\begin{abstract}
{\lipsum[1-4]}
\end{abstract}

\begin{dedicatoria} % opcional
Una dedicatoria corta. Por ejemplo, \emph{A los creadores de U-Campus}
\end{dedicatoria}

\begin{thanks} % opcional
\lipsum[1-2]
\end{thanks}
\cleardoublepage

\tableofcontents
\listoftables % opcional
\listoffigures % opcional

\mainmatter

\begin{intro}

	La protección de la confidencialidad de la información manipulada por los programas computacionales es un problema cuya relevancia se ha incrementado en el último tiempo, a pesar de tener varias décadas de investigación. Por ejemplo, una aplicación web (o móvil) que como parte de su funcionamiento debe interactuar con servicios de terceros y por tanto debe proteger que su información sensible no se escape durante la ejecución de la aplicación a canales públicos.

	Muchas de las técnicas de seguridad convencionales como \textit{control de acceso} tienen deficiencias para proteger la confidencialidad de un programa, por ejemplo no restringen la propagación de información\cite{myers-phd}.

	Formas más expresivas y efectivas de proteger la confidencialidad se basan en un análisis estático sobre el código del programa, y se categorizan dentro de \textit{language-based security}. Una de las técnicas más efectivas se denomina \textit{tipado de seguridad} en un \textit{lenguaje de seguridad}, donde los tipos son anotados con niveles de seguridad para clasificar la información manipulada por el programa.

	Uno de los mayores desafíos de los lenguajes de seguridad es facilitar el trabajo del programador, utilizando técnicas más expresivas. En esta dirección, Cruz et al.~\cite{cruzAl:ecoop2017} recientemente propusieron \textit{type-based declassification}, una variación de tipado de seguridad que utiliza el sistema de tipos del lenguaje para controlar la declasificación de la información.

	\textit{Type-based declassification} presenta limitaciones en cuanto a su implementación, debido a que el análisis teórico se realiza sobre un lenguaje minimalista que no incluye características básicas de los lenguajes de programación, como instrucciones condicionales y mutabilidad. En este trabajo, se propone una extensión al lenguaje minimalista utilizado en \textit{type-based declassification}, y la implementación en el lenguaje de programación Dart, desarrollando un plugin para los editores de texto más populares, con el objetivo de proporcionar una experiencia interactiva e intuitiva al usuario.



		\section{Motivación}
	El tema escogido se considera interesante debido a su importancia teórica y práctica en el campo de la seguridad en lenguajes de programación. Los fundamentos teorícos de \textit{type-based declassification} están bien descritos~\cite{cruzAl:ecoop2017}, pero no así su realización práctica, siendo esto reconocido por sus autores.

	Se considera que la realización de este trabajo es relevante para demostrar y materializar en un lenguaje de uso general la investigación de type-based declassification. Dicha materalización constituye un desafío importante en términos de complejidad, al tener que entender la teoría subyacente de tipado de seguridad, así como extender un lenguaje real y sus herramientas de análisis estático con nuevas funcionalidades.

		\section{Objetivos}
	El objetivo de la memoria es realizar la implementación de un sistema de inferencia para \textit{type-based declassification}. Dentro de los objetivos específicos del trabajo, podemos encontrar:

	\begin{itemize}
	\item \textbf{Inferencia y verificación estática de type-based declassification}. Se entiende coma la implementación de un sistema de inferencia de facetas de declasificación para \textit{type-based declassification}, en el lenguaje de programación Dart. Dentro del análisis de la inferencia se incluye la verificación de las reglas del sistema de tipos de \textit{type-based declassification}.

	\item \textbf{Plugin para editores}. Mostrar al programador el resultado de la inferencia, por medio de un plugin para los editores de texto que soporten servidores de análisis estático de Dart, ofreciéndole acciones al respecto.

	\end{itemize}

		\section{Resultados y Organización del documento}

	En términos concretos este trabajo presenta el diseño de un plugin que realiza el análisis de type-based declassification. Los antecedentes teóricos necesarios para entender este trabajo se abordan en el capítulo 2, mientras que la propuesta de solución es desarrollada en el capítulo 3.

	Los detalles de diseño de implementación de la propuesta son revisados en el capítulo 4.

	% TODO
	Por terminar.


\end{intro}

\chapter{Marco Teórico}

	La protección de la confidencialidad de la información manipulada por los programas computacionales es un problema cuya relevancia se ha incrementado en el último tiempo, a pesar de tener varias décadas de investigación. Por ejemplo, una aplicación web (o móvil) que como parte de su funcionamiento debe interactuar con servicios de terceros y por tanto debe proteger que su información sensible no se escape durante la ejecución de la aplicación a canales públicos.

\section{Information flow control}

	Muchas de las técnicas de seguridad convencionales como \textit{control de acceso} tienen deficiencias para proteger la confidencialidad de un programa, por ejemplo no restringen la propagación de información [].% TODO citation needed

	Un mecanismo más expresivo para la protección de la confidencialidad e integridad de la información se denomina \textit{information flow control}. Mientras control de acceso restringe qué datos pueden ser accedidos, \textit{information flow control} restringe el flujo de estos datos.

	\subsection{Tipado de seguridad}
	El análisis de \textit{information flow control} puede ser realizado sobre el código del programa, de forma estática o dinámica. Una de las técnicas más efectivas de \textit{information flow control} con análisis estático es \textit{tipado de seguridad} en un \textit{lenguaje de seguridad}. En un lenguaje de seguridad, los valores y los tipos son anotados con niveles de seguridad para clasificar la información que el programa manipula. Dichos niveles de seguridad forman una \textit{lattice}\footnote{Un orden parcial, donde todo par de elementos tiene un único supremo e ínfimo}. Por ejemplo con la \textit{lattice} de dos niveles de seguridad $L \sqsubseteq H$, se puede distinguir entre valores enteros públicos o de baja confidencialidad ($Int_L$) y valores enteros privados o de alta confidencialidad ($Int_H$). El sistema de tipos usa estos niveles de seguridad para prevenir que la información confidencial no fluya directa o indirectamente hacia canales públicos []. % TODO citation needed

	\subsection{No-interferencia}
	Formalmente, la propiedad de confidencialidad puede ser expresada como una propiedad de \textit{no-interferencia (noninterference)}. A grandes rasgos noninterference expresa que para dos ejecuciones realizadas por el adversario de un programa seguro, con valores confidenciales equivalentes, las salidas del programa deben ser equivalentes para el adversario. Esto caracteriza que el adversario no aprende nada sobre los valores confidenciales de un programa.

	El siguiente programa ilustra el concepto de noninterference. Este muestra un método \texttt{login} para verificar la contraseña de un usuario. Se considera que el parámetro \texttt{password} es privado (y por tanto no provisto por el adversario), mientras que el parámetro \texttt{guess} es público (y lo provee el adversario).

	\begin{lstlisting}
String login(String password, String guess) {
	if (password == guess) return "Login Successful";
	else return "Login failed";
}
	\end{lstlisting}

	Este programa no cumple noninterference, pues el adversario puede aprender sobre la variable confidencial \texttt{password} observando el valor de retorno del método para distintas ejecuciones.

	\subsection{Declasificación}
	En una aplicación real y práctica deseamos que el programa anterior sea aceptado a pesar de violar la propiedad de no-interferencia, pues de otra forma no tendríamos cómo realizar la autenticación. Para solucionar este problema, los lenguajes de seguridad adicionan mecanismos para \textit{declasificar} la información confidencial, implementados de diferentes formas []. Una de ellas, por ejemplo en Jif (un lenguaje de seguridad) [] es usar un operador \texttt{declassify}, como se indica en el siguiente ejemplo, declasificando la comparación de igualdad del parámetro confidencial \texttt{password} con el parámetro público \texttt{guess} %TODO citation needed.

\begin{lstlisting}
String login(String password, String guess) {
	if (declassify(password == guess)) return "Login Successful";
	else return "Login failed";
}
\end{lstlisting}

	Esto no corresponde a una amenaza de seguridad, debido a que el resultado de la operación de comparación es negligible con respecto al parámetro privado \texttt{password}. Sin embargo, usos arbitrarios del operador \texttt{declassify} pueden resultar en serias fugas de información, como por ejemplo \texttt{declassify(password)}.

	\section{Type-based declassification}

	Varios mecanismos se han explorado para controlar el uso de declasificación, y poder asegurar además una propiedad de seguridad para el programa []. En esta dirección, Cruz et al. [] recientemente propusieron \textit{type-based declassification} como un mecanismo de declasificación que conecta la abstracción de tipos con una forma controlada de declasificación, en una manera intuitiva y expresiva, proveyendo garantías formales sobre la seguridad del programa. %TODO citation needed.

	\subsection{Sistema de tipos}

	En \textit{type-based declassification} los tipos tienen dos facetas, una que refleja el tipo de implementación y otro tipo que refleja las operaciones de declasificación sobre los valores de dicho tipo. Por ejemplo, el tipo $\text{StringEq} \triangleq [\text{eq} : \text{String} \rightarrow \text{Bool}]$ autoriza la operación \texttt{eq} sobre un String. Entonces se puede usar el tipo de dos facetas $\text{String} < \text{StringEq}$, en donde String es un subtipo de StringEq, para controlar la operación de declasificación de la igualdad sobre \texttt{password}.

	\begin{lstlisting}
String login(String<StringEq password, String guess) {
	if (password.eq(guess)) return "Login successful";
	else return "Login failed";
}
	\end{lstlisting}

	Para formalizar y demostrar propiedades del sistema de tipos, se utilizó el lenguaje $\text{Ob}_{\text{SEC}}$, que se muestra en la figura. % TODO poner lenguaje Ob_SEC en figura

	$\text{Ob}_{\text{SEC}}$ es un lenguaje minimalista, y por tanto no soporta características comunes de lenguajes de programación, como asignaciones a variables y condicionales. La justificación de su uso se debe a que era suficiente para formular y demostrar la proposición.

	\subsubsection{Características}

	Se demostró que programas de $\text{Ob}_{\text{SEC}}$ bien tipados son \textit{safe}\footnote{Sin errores de tipos en tiempo de ejecución}.

	\subsection{Relaxed noninterference}

	La propiedad de seguridad que se demuestra para $\text{Ob}_{\text{SEC}}$ es una forma de no-interferencia con políticas de declasificación, denominada \textit{Relaxed noninterference}.	Un lenguaje de seguridad que tiene esta propiedad, garantiza que la información confidencial solo puede fluir hacia canales públicos de una forma controlada, por medio de las políticas de declasificación. % TODO citation needed

	\section{Lenguaje Dart}

	Dart [] es un lenguaje de programación de propósito general, orientado a objetos y de código abierto desarrollado por Google. Es usado para construir aplicaciones web, móviles y dispositivos IoT (Internet of Things). % TODO citation needed

	Dart es un lenguaje de creciente adopción. Actualmente, la interfaz de Google Adwords está construida sobre Dart y Angular2. Además, Dart es usado en el framework de desarollo multiplataforma \textit{Flutter}.

	El lenguaje Dart fue escogido porque los investigadores que realizaron el trabajo de \textit{type-based declassification} estudian este lenguaje como parte de un proyecto de investigación mayor en el área de seguridad. Es factible implementar el enfoque de \textit{type-based declassification} en Dart, dado que fue formalizado considerando un lenguaje minimalista orientado a objetos ($\text{Ob}_{\text{SEC}}$).

\chapter{Especificación del problema}

\chapter{Primero}
\section{Hola}
\lipsum[1-3]
\begin{defn}[ver \cite{KAR00}] Definición definitiva $$\frac{d}{dx}\int_a^xf(y)dy=f(x).$$\end{defn}

\chapter{Segundo}
\lipsum[50-60]
\begin{conclusion}

	Type-based declassification muestra una conexión entre las relaciones de subtyping de un lenguaje orientado a objetos, y las relaciones de orden que conforman los tipos de seguridad, para proponer un sistema de tipos que cumple una versión relajada de noninterference. Con esta propuesta, Cruz \textit{et al.} abordan parcialmente el desafió de integrar los modelos de control de flujo de información con infraestructuras existentes. Este trabajo materializa aquella propuesta, con una implementación para un subconjunto del lenguaje Dart, en conjunto con un sistema de inferencia y un plugin para editores.

	A pesar del foco de seguridad que tiene un trabajo de estas características, la formulación del problema de inferencia y el uso del plugin para integrar los resultados fueron concebidos teniendo al programador en mente, para facilitarle el trabajo y mejorar su experiencia programando, entre otros beneficios. Esta experiencia puede mejorar aún más, agregando nuevas características al plugin.

	\section*{Trabajo futuro}

	\paragraph{Formalización de inferencia.}En este trabajo se implmentó un sistema de inferencia sin demostrar que la extensión necesaria al sistema de tipos de type-based declassification preserva las propiedades del trabajo original, como son \emph{type safety} y relaxed noninterference. En este sentido, es deseable la formalización de la inferencia de tipos de dos facetas, antes de realizar cualquier extensión a este trabajo, cuyo objetivo fue demostrar el uso práctico del enfoque propuesto por Cruz \textit{et al.}

	\paragraph{Extensión al subconjunto soportado.}Este trabajo soporta un subconjunto pequeño del lenguaje Dart, lo que no permite probarlo en aplicaciones reales de mayor envergadura. Soportar características avanzadas del lenguaje, e implementar la herramienta en otros lenguajes de programación, permitiría posicionar a la herramienta como una alternativa competente de análisis de control de flujo para aplicaciones en producción.

	\paragraph{Características del plugin.}El plugin implementado en este trabajo solo muestra los resultados de la inferencia, pero no permite al programador tomar acciones automáticas al respecto. Por ejemplo, es posible asistir al usuario en la definición de una faceta pública que ha sido declarada, navegar al lugar donde se define una faceta pública al ubicarse en la faceta declarada, definir y declarar una faceta pública basándose en el resultado de la inferencia, entre otros.



\end{conclusion}


% \input{glosario.tex} % opcional

\bibliographystyle{plain}
\bibliography{bibliografia}

% \input{anexo_apendices.tex} % opcionales

\end{document}
